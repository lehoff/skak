\message{ !name(longmoves.tex)}\documentclass[11pt,a4paper]{book}
%

\usepackage{skak}
%\usepackage{texmate}
%\usepackage{chessboard}
%


\begin{document}

\message{ !name(longmoves.tex) !offset(-3) }

\pagestyle{headings}

\styleC

\chapter*{}

\section*{8.\ partija\\
DAMIN GAMBIT. TARRASCHEVA OBRAMBA}
{\tiny\bf Igrana 26. julija 1906. leta v Nurnbergu}
%%
%\newchessgame[id=Vidmar_8,
%white={M. Vidmar},
%black={dr. S. Tarrasch},
%result={1-0}]
%%
%\begin{center}
%{\em\xskakgetgame{white}\ ---\ \xskakgetgame{black}}
%\end{center}

\newgame\longmoves
\mainline{1.d4 d5 2.c4 e6 3.Nc3 c5}
%

basuhsa 
(\variation{3... Nf6})
nezadostna in da rni proti daminemu
gambitu ne more dovolj zgodaj igrati \wmove{c7—c5}. To obrambno potezo je poskušal celo kot odgovor na 2. c2—c4, potem pa se je z njo utaboril
v tretjem poteznem paru. Njegov veliki sloves je njegovi obrambi dajal 
odgovarjal
\variation{4.e3}.
Nazadnje pa je A. Rubinstein n pravo pot:
\variation{4. cxd5 exd5 5. Nf3 Nf6 6. g3 Nc6 7. Bg2 Be7 8.O-O O-O}.
V tej varianti ima črni nedvomno teave, ki so gotovo nekoliko veje od teav v
ortodoksni obrambi dami-nega gambita. Toda Tarrasch jih nikoli ni priznal. Tarrasch tudi ni nikoli popolnoma razumel svojega velikega
učenca Rubinsteina. Ta veliki mojster je bil namreč nesporno eden izmed vodilnih pionirjev hipermoderne ole, in Reti ter Nimcovi,
priznana ustanovitelja te ole, mu ne moreta krajšati ustreznih zaslug.

\end{document}
\message{ !name(longmoves.tex) !offset(-52) }
