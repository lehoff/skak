\documentclass[10pt]{article}

\usepackage{lambda,ifthen,calc}
\usepackage{tabularx}
\usepackage[ps,mover,small,skaknew]{skak}
\usepackage{fancyvrb}
\usepackage{multicol}
\usepackage[final]{showexpl}



\makeatletter
%% to get showexpl to respekt empty lines
%% this corrects a bug in showexpl as long
%% as the new version isn't there
\renewcommand*\SX@resultInput{%
  %%\MakePercentComment\catcode`\^^M=10\relax
  \SX@@preset\SX@preset
  \if@SX@rangeaccept
    \let\@tempa=\SX@input% Nur sinnvoll bei \LTXinputExample
  \else
    \let\@tempa=\input
  \fi
  \@tempa{\SX@codefile}%
  \MakePercentIgnore}
\makeatother


\lstset{width=0.5} % wider examples
\lstset{preset=\raggedright}

\usepackage[inactive]{pst-pdf}
%\usepackage{pst-pdf}

\makeatletter
% to be able to redefine showboard, should go in skak.sty
\def\notationOn{\let\print@board=\show@board@notation%
  \let\print@inverseboard=\show@board@notation@inverse}
\def\notationOff{\let\print@board=\show@board%
  \let\print@inverseboard=\show@board@inverse}

\renewcommand\showboard{\print@board}
\renewcommand\showinverseboard{\print@inverseboard}
\makeatother

\notationOn

\let\ORIshowboard\showboard
\let\ORIshowinverseboard\showinverseboard
\renewcommand\showboard{%
    \makebox[8\squarelength]{%
    \rule{0pt}{9\squarelength}% for better spacing above board
    \begin{postscript}% siehe pst-pdf.sty
    [trim = \squarelength{} 0pt \squarelength{} 0pt]
    \ORIshowboard
    \end{postscript}}}

\renewcommand\showinverseboard{%
    \makebox[8\squarelength]{%
    \rule{0pt}{9\squarelength}% for better spacing above board
    \begin{postscript}% siehe pst-pdf.sty
    [trim = \squarelength{} 0pt \squarelength{} 0pt]
    \ORIshowinverseboard
    \end{postscript}}}


\newcommand{\package}[1]{\textsf{#1}}
\newcommand{\command}[1]{\textbackslash\texttt{#1}}
\newcommand{\pgn}{PGN\ }
\newcommand{\san}{SAN\ }
\newcommand{\fen}{FEN\ }
\newcommand{\xboard}{xboard}
\newcommand{\metafont}{\textsc{Metafont}\ }
\newcommand{\filename}[1]{\texttt{#1}}
%\newenvironment{example}
%   {\VerbatimEnvironment\begin{VerbatimOut}{test.out}}
%   {\end{VerbatimOut}\small\noindent
%\begin{multicols}{2}
%\BVerbatimInput{test.out}
%\columnbreak%
%
%\input{test.out}%
%\end{multicols}
%   \normalsize}



\newcommand{\Guido}{Guido Governatori\ }
\newcommand{\Harri}{Harri Haanpaa\ }
\newcommand{\Ingo}{Ingo Sander\ }
\newcommand{\Dirk}{Dirk B\"achle\ }
\newcommand{\Ulrike}{Ulrike Fischer\ }
\newcommand{\Andreas}{Andreas Wilm\ }



\title{Typesetting Chess in \LaTeX with the \package{skak} Package\\
Version 1.5.2}
\author{Torben Hoffmann\\ e-mail: \texttt{torben.lehoff@gmail.com}}
%%%%%%%%%%%%%%%%%%%%%%%%%%%%%%%%%%%%%%%%%%%%%%%%%%%%%%%%%%%%%%%%%%%%%%%%%%%%%%%%
\begin{document}
\maketitle

\begin{abstract}
  This user guide shows how to use an enhancement to the existing
  package for typesetting chess in \LaTeX (\package{skak}). The
  primary improvement over the old package is that it has become
  easier to typeset chess games with this new package.
\end{abstract}

\newpage
\tableofcontents
\newpage



%%%%%%%%%%%%%%%%%%%%%%%%%%%%%%%%%%%%%%%%%%%%%%%%%%%%%%%%%%%%%%%%%%
\section{Terms of Usage}
\label{sec:terms-usage}

This package is distributed under the terms described in the Latex
Project Public Licence, i.e.,

\begin{quote}
  This software is copyright but you are granted a license which gives
  you, the ``user'' of the software, legal permission to copy,
  distribute, and/or modify the software. However, if you modify the
  software and then distribute it (even just locally) you must change
  the name of the software to avoid confusion.
\end{quote}



%%%%%%%%%%%%%%%%%%%%%%%%%%%%%%%%%%%%%%%%%%%%%%%%%%%%%%%%%%%%%%%%%%
\section{Installing the \package{skak} Package}
\label{sec:install}

I could not get the makefile to do the job, so if someone could help
me out on this I would be very glad indeed.

Basically all you have to do is to follow the guidelines in the
makefile. First you create all the required directories (stated in the
\texttt{install} section of the makefile) and then you run the
\texttt{make install} command. Remember to run \texttt{texhash} after
you have updated your texmf tree.

Sorry that it is not working $100\%$--- I just do not have the time for
it right now.

If you cannot get the things working just put the files where \TeX and
the other tools can find them and you will be just fine.

For Windows user I can recommend the MikTeX installation since it has
skak as a package and all you have to do is to use the MikTeX
installation tools and select installation of skak and you're up and
running.

Ultimately the skak package should be documented using the \LaTeX
standard documentation, but that is a major undertaking that will not
happen right now.

\subsection{Using the \package{lambda} Package}
\label{sec:using-lambda}

The \package{skak} package uses the \package{lambda} package to do
some of the hard processing necessary to allow the \pgn notation as
input --- since it is not standard in \LaTeX distributions it is
included in the distribution of the \package{skak} package.


%%%%%%%%%%%%%%%%%%%%%%%%%%%%%%%%%%%%%%%%%%%%%%%%%%%%%%%%%%%%%%%%%%%%%
\section{Basic usage of the \package{skak} package}
\label{sec:basic-usage}

Once the \package{skak} package has been installed you only need to
know very few things in order to produce nicely typeset chess games:

\begin{itemize}
\item write \verb|\usepackage{skak}| in the preamble.
\item the \command{newgame} command.
\item the \command{mainline} command.
\item the \command{variation} command.
\item the \command{showboard} command.
\end{itemize}

As an example I will describe some moves from the Ruy Lopez opening
--- please note that the comments are just idle talk. In the left
column the input is shown and the corresponding output is shown in the
right column.

\begin{LTXexample}
\newgame
\mainline{1.e4 e5 2. Nf3 Nc6 3.Bb5}

\showboard
\end{LTXexample}

The \command{newgame} command sets up the internal board or a new
game. The \command{mainline} command does two things: it typesets the
moves and updates the board so that \command{showboard} will show the
current board position.

\begin{LTXexample}
\mainline{3...a6} \savegame{pap1}
Rarely used these days,
\variation{3...Nf6} is more in
fashion since blah blah...
\mainline{4.Ba4}

\showboard
\end{LTXexample}



This example shows that if it is black to move the move number shoud
be followed by three periods. Furthermore, it also shows that the
\command{variation} command merely typesets the moves --- it does
\emph{not} update the board, but apart from that it expects the same
input as \command{mainline}.

The four commands introduced in this section will enable you to
comment your games with minimal effort and still get great results ---
in fact most chess books can be writen using only these commands!

But the \package{skak} package offers a lot more that can make chess
typesetting easier for your --- these features are the topics of the
next two sections.



%%%%%%%%%%%%%%%%%%%%%%%%%%%%%%%%%%%%%%%%%%%%%%%%%%%%%%%%%%%%%%%%%%%%%%
\section{Intermediate Usage of the \package{skak} Package}
\label{sec:intermediate}

This section describes how to
\begin{itemize}
\item do game management,
\item add ornaments to the board and
\item controlling some aspects of showing the board.
\end{itemize}


\subsection{Game Management}
\label{sec:game-management}

Returning to the Ruy Lopez example suppose you want to go into details
about a number of variations after white's third move, preferably
showing the board in all the lines you describe.

One way to do this is to use the \command{hidemoves} command like
this:

\begin{LTXexample}
\newgame
\mainline{1.e4 e5 2. Nf3 Nc6 3.Bb5}

\showboard
\end{LTXexample}

\begin{LTXexample}
\mainline{3...a6} blah blah

\showboard

\end{LTXexample}

\begin{LTXexample}
\newgame
\hidemoves{1.e4 e5 2. Nf3 Nc6 3.Bb5}
\mainline{3...Nge7} blah blah

\showboard
\end{LTXexample}

This is, admittedly, a labourious way of doing it, but the example
shows how \command{hidemoves} can fast-forward you to any point in a game.


\package{skak} offers you ways of storing game positions
\begin{itemize}
\item in memory or
\item in a file.
\end{itemize}

The former is done using the \command{storegame} and
\command{restoregame} commands like this:

\begin{LTXexample}
\newgame
\mainline{1.e4 e5 2. Nf3 Nc6 3.Bb5}
\storegame{game1}
\mainline{3...a6 4.Ba4 b5}

\showboard
\restoregame{game1}

\mainline{3...Bd7}
\showboard
\end{LTXexample}


Saving and loading to/from a file is the same just with the commands
\command{savegame} and \command{loadgame}.


\section{Ornaments}
\label{sec:ornaments}



\subsection{Showing the Board}
\label{sec:showing-board}

We have already seen \command{showboard} in action, but is is possible
to alter the way the board looks like.

With the \command{mover}\footnote{\command{mover} only works if the \package{skak} is loaded with the ps option.} command one can request that a little marker
is shown next to the board indicating who is to move next.

\begin{LTXexample}
\newgame

\showboard

\end{LTXexample}

\begin{LTXexample}%9 okay

\mainline{1.d4}

\showboard
\end{LTXexample}

This can be turned off with \command{moveroff}. You can also remove
the notation around the board using \command{notationoff} which gives
this appearance:

\begin{LTXexample}
\newgame
\notationoff
\showboard

\end{LTXexample}

The notation is added again after one has issued
\command{notationon}. \notationon

A more elaborate explanation of how the appearance of the notation
around the board is configured can be found in \cite{chessnfss}.

Although it is rarely seen is chess books \package{skak} offers
the possibility of showing the board from black's perspective
using the
\command{showinverseboard} command.

\begin{LTXexample}
\newgame
\showboard

\end{LTXexample}

If you are writing about an opening from black's point of view this
command/feature could be a welcome refreshment in presentation style.


\subsection{Handy Commands}
\label{sec:handycommands}

By requst of users and own usage a couple of handy commands have been
added to \package{skak} that you might find useful too.

\command{wmove\{\emph{move}\}} typesets a single move, whereas
\command{bmove\{\emph{move}\}} does the same only with leading
ellipses (see \ref{sec:advanced-usage}for in-depth explanation). This
is useful for short remarks like:

\begin{LTXexample}
\bmove{Nd4} would have been better.
\end{LTXexample}


\command{lastmove} is a shorthand for commenting on the last move made:

\begin{LTXexample}
\newgame
\mainline{1.a3}

\lastmove{} is bad.
\end{LTXexample}

%%%%%%%%%%%%%%%%%%%%%%%%%%%%%%%%%%%%%%%%%%%%%%%%%%%%%%%%%%%%%%%%%%%%%%%%%%%%%
\section{Advanced Usage}
\label{sec:advanced-usage}

This section will describe some features of the \package{skak} package
that allows you to control a lot with regards to the typesetting of
moves as well as how the board is shown.

Furthermore, we will take a look at how the board is set up in
arbitrary positions and how the input language can be changed.


\subsection{Tailoring of the Board Appearance}
\label{sec:tail-board-appe}

In additon to the ornaments and showing the board upside down
described in \ref{sec:intermediate} one can choose to show only a
subset of the pieces actually on the board.

The simple commands \command{showonlywhite} and \command{showonlyblack} shows
only one colour on the board:

\begin{LTXexample}
\newgame
\mainline{1.e4 e5 2. Nf3 Nc6 3.Bb5}

\showonlywhite
\showboard

\end{LTXexample}

\begin{LTXexample}
\showonlyblack
\showboard
\end{LTXexample}


This is often used when describing openingswhere one wants to focus on
the merrits of the piece set-up on one side without considering how
the other side is positioned.

In addition to the commands for showing only one colour it is also
possible to show any subset of the pieces on the board using the
\command{showonly} command that takes a comma sperated list of piece
names as argument:

\begin{LTXexample}
The pawn structure is
now like this:

\showonly{P,p}
\showboard
\end{LTXexample}

The piece names are the standard English name with the addition that
white pieces are denoted by upper case and black's by lowercase. Pawns
are P and p.

It is also possible to do the opposite of \command{showonly} using
\command{showallbut}:

\begin{LTXexample}
\showallbut{P,p}
\showboard
\end{LTXexample}


\subsection{Setting up the Board}
\label{sec:setting-up-board}

So far we have only looked how to start a new game, but you can also
set up the board in \emph{any} position using the command
\command{fenboard}:

Forsyth-Edwards Notation (FEN) is a standard used to describe chess
positions. It starts with a description of the contents of the ranks
on the board. The ranks are separted by a slash starting with the
eighth row, so if you want to write down the FEN of a
position you should be seated as white and work your way dow the
board. The pieces are designated using the stardard English piece
names.

After the position of the pieces the following information is given
separated by spaces:
\begin{itemize}
\item who it is to move (w or b).
\item castling options left for white. - for none, and K and Q for
  options to castle kingside or queenside for white and k and q for black
\item en passant square, if applicable otherwise a -.
\item number of halfmoves since the last capture.
\item the move number.
\end{itemize}


\begin{LTXexample}
\fenboard{r5k1/1b1p1ppp/p7/1p1Q4/
2p1r3/PP4Pq/BBP2b1P/R4R1K w - - 0 20}

\mbox{}
\bigskip

\showboard


\mainline{20.Qxb7 Rae8 21.Qd5}

\end{LTXexample}


\subsection{Typesetting Style}
\label{sec:typesetting-style}

There are a couple of ways to configure how the move are typeset as
described in the following.


\subsubsection{\command{styleA}}
\label{sec:stylea}

The \command{styleA} will switch the typesetting to the way it has
been the standard in Denmark and --- I believe --- Germany.

\begin{LTXexample}
\newgame
\styleA

\mainline{1.e4 c5 2.Nf3 Nc6} 
\end{LTXexample}


\subsubsection{\command{styleB}}
\label{sec:styleb}

The style selected by \command{styleB} is the default one.

\begin{LTXexample}
\newgame
\styleB

\mainline{1.e4 c5 2.Nf3 Nc6} 
\end{LTXexample}


\subsubsection{\command{styleC}}
\label{sec:styleb}

The style selected by \command{styleC} is an elaboration on
\command{styleB}: it typesets the moves line by line for each move for
the mainline but on the same line for variations --- this style is
used in many books these days, but look the best in a two-column
setting where the linewidth is small.

\begin{LTXexample}
\newgame
\styleC

\mainline{1.e4 c5 2.Nf3 Nc6} 

Interesting how \variation{2...d6} is blah blah.
\end{LTXexample}

\styleB


\subsubsection{\command{variationstyle} and  \command{mainlinestyle}}
\label{sec:variationstyle}

Configuration of these commands allows for total control of the
typesetting of the mainline and variations, they default to:

\begin{verbatim}
\def\mainlinestyle{\bfseries}% could also contain
                             % definitions of the
                             % various style options
\def\variationstyle{}% as with mainlinestyle
\end{verbatim}

They are used by \command{styleC} to control the switching back and
forth between a tabbed environment and normal typesetting:

\begin{verbatim}
\def\mainlinestyle{\styleC\bfseries}
\def\variationstyle{\styleB}
\end{verbatim}



\subsubsection{\command{longmoves} and \command{sanmoves}}
\label{sec:commandlongmove}

The \command{longmoves} command will ensure that the full move, i.e., including the from
square is typeset. This is also know as \emph{algebraic} notation ---
can be quite handy when writing about chess for children or other
newcommers to chess notation.

\begin{LTXexample}
\newgame
\longmoves

\mainline{1.e4 c5 2.Nf3 Nc6} 
\end{LTXexample}

\shortmoves

The \command{shortmoves} command returns things to normal.


\subsection{Input Language}
\label{sec:input-language}

The \package{skak} package by default uses the English letters for the
pieces, i.e., K for King, Q for Queen, R for Rook, B for Bishop, N for
Knight and P for Pawn.

The \command{newskaklanguage} command makes it possible to define
other languages than English, e.g., suppose I want to use the Danish
piece letters:

\begin{LTXexample}
\newskaklanguage{danish}{KDTLSB}

\skaklanguage[danish]

\newgame
\mainline{1.e4 c5 2.Sf3 Sc6} 
\end{LTXexample}



\end{document}
