\documentclass[10pt,a4paper,twocolumn,landscape]{article}

%\usepackage{a4wide}
\usepackage[dvips]{geometry}

\usepackage{titlesec}
\titleformat{\section}
  {\sffamily\large\bfseries}{}{0em}{}
\titleformat{\subsection}
  {\sffamily\normalsize\bfseries}{}{0em}{}

\titlespacing*{\section}{0pt}{1.5ex plus 0.5ex minus .2ex}{1ex plus
  .2ex}
% \titlespacing*{\subsection}{0pt}{2.5ex plus 0.75ex minus .2ex}{1ex
%   plus .2ex}

\usepackage{fancyhdr}
\pagestyle{fancy}
\fancyhead{}
\chead{\textbf{\texttt{skak} version 1.5 \textsf{Reference Manual}}}


\usepackage[ps]{skak}

\newcommand{\Cmd}[1]{\textsf{\textbf{\textbackslash#1}}}
\newcommand{\Arg}[1]{\textsf{\textsl{#1}}}

\newcommand{\Option}[2]{\textsf{\textbf{#1}}\ #2\smallskip}

\newcommand{\ArgInDescription}[1]{\Arg{\{#1\}}}

\newcommand{\command}[3]{\Cmd{#1}\Apply{\ArgInDescription}{\Listize[#2]}\newline#3\smallskip}

\newcommand{\simpleCommand}[2]{\Cmd{#1}\newline#2\smallskip}

\setlength{\parindent}{0pt}

\newcommand{\myheading}[1]{{\large{\textsf{\textbf{#1}}}}\bigskip}

\begin{document}
%\tracingmacros=0

%\twocolumn

%%%%%%%%%%%%%%%%%%%%%%%%%%%%%%%%%%%%%%%%%%%%%%%%%%%%%%%%%%%%%%%%%%
\section{Updating the board}
%\myheading{Updating the board}

% These commands are used to update the position on the
% board. \Cmd{mainline} is the heart of the \texttt{skak} package.



\simpleCommand{newgame}
{Initialises the board to the opening position.}

\command{mainline}{SAN moves}
{Updates the board with the \Arg{SAN moves} and
typesets \Arg{SAN moves} according to the current \Cmd{mainlinestyle}.}

\command{hidemoves}{SAN moves}
{Updates the board with the \Arg{SAN moves} but does
  \emph{not} typeset the moves --- this is useful for commenting a
  game where you want to focus on a certain position after
  some moves have already been made.}

\def\tmpCmd#1{\Arg{\textless row#1\textgreater/}} 
\command{fenboard}{FEN position}
{Initialises the board to the position described with \Arg{FEN
    position}. The format of a FEN position is:\newline
  \Arg{\textless board rows\textgreater} \Arg{w\textbar
  b} \Arg{\textless castling options\textgreater} \Arg{\textless en
  passant square\textgreater}\newline \Arg{\textless 50 moves
  counter\textgreater} \Arg{\textless move number\textgreater}\newline
The FEN for the opening position is\newline 
\Arg{rnbqkbnr/pppppppp/8/8/8/8/PPPPPPPP/RNBQKBNR}\newline
\Arg{w KQkq - 0 1}\newline
Note: the 50 moves counter is not used by the \texttt{skak} game
engine, but it is updated to stay in sync with external programs.}

% \simpleCommand{boardasfen}
% {Returns the current position in the FEN notation.}

%\newpage



%\newpage
%%%%%%%%%%%%%%%%%%%%%%%%%%%%%%%%%%%%%%%%%%%%%%%%%%%%%%%%%%%%%%%%%%
\section{Describing moves}
\bigskip


% Apart from showing the moves that update the board it is also
% possible to describe moves in the running text.  

\command{variation}{SAN moves}
{This will typeset \Arg{SAN moves} using the current \Cmd{variationstyle}. This command undoes the last move so you have to start one ply back.}

\command{variationcurrentt}{SAN moves}
{Like \Cmd{variation}, but does not undo the last move.}

\command{continuevariation}{SAN moves}
{Continues the variation but undoes the last move first.}

\command{continuevariationcurrent}{SAN moves}
{Continues the variation without undoing the last move first.}

\command{wmove}{SAN move}
{Typesets \Arg{SAN move} using the current \Cmd{variationstyle}.
Example: \texttt{\textbackslash wmove\{Nf3\}} gives \wmove{Nf3}.}

\command{bmove}{SAN move}
{Typesets \Arg{SAN move} using the current \Cmd{variationstyle} but
  with \ldots (or something similar according to the style) in front
  of the move --- can be used to describe a single black move. Example:
 \texttt{\textbackslash bmove\{Nxd4\}} gives \bmove{Nxd4}}

\command{movecomment}{Chess moves}
{Typesets the \Arg{Chess moves} using the current \Cmd{variationstyle},
but doesn't check for move numbers like \Cmd{variation} does.}

%%%%%%%%%%%%%%%%%%%%%%%%%%%%%%%%%%%%%%%%%%%%%%%%%%%%%%%%%%%%%%%%%%
\section{Showing the board}



\simpleCommand{showboard}
{Shows the current position from whites perspective.}

\simpleCommand{showinverseboard}
{Shows the current position from blacks perspective.}



\section{Style selection}

\simpleCommand{styleA}
{Chooses the \Arg{styleA} for typesetting of moves.}

\simpleCommand{styleB}
{Chooses the \Arg{styleB} for typesetting of moves. This is the
  default style.}

\simpleCommand{styleC}
{Chooses the \Arg{styleC} for typesetting of moves.}


%\newpage


\section{Size of the board}



\simpleCommand{normalboard}
{The default size of the board typeset by the \Cmd{showboard}
  commands.}

\newcommand{\boardsize}[1]{
\simpleCommand{#1board}
{The \Cmd{showboard} commands will be typeset in a #1 font.}}

\boardsize{tiny}

\boardsize{small}

\boardsize{large}

%\newpage
\section{Notation and mover}

\simpleCommand{notationOn}
{The \Cmd{showboard} commands show rank and file names. This is the
  default.}

\simpleCommand{notationOff}
{The \Cmd{showboard} commands show only the board.}

\simpleCommand{showmoverOn}
{The \Cmd{showboard} commands indicate --- with a small box --- which
  player has to move. Note: this only works when the ps option is used.}

\simpleCommand{showmoverOff}
{The dual of \Cmd{showmoverOn}.}



\section{Selective showing of pieces}

\simpleCommand{showall}
{Makes the \Cmd{showboard} commands show all pieces.}

\simpleCommand{showonlywhite}
{The \Cmd{showboard} commands will only show the white pieces.}

\simpleCommand{showonlyblack}
{The \Cmd{showboard} commands will only show the black pieces.}

\simpleCommand{showonlypawns}
{The \Cmd{showboard} commands will only show the pawns.}

\command{showonly}{piece names}
{The argument \Arg{piece names} is a comma separated list of names of
  pieces to be shown using the \Cmd{showboard} commands. White pieces
  are named \Arg{K,Q,R,B,N,P} and black's \Arg{k,q,r,b,n,p}. Note: called with no arguments all pieces are showed!}

\command{showallbut}{piece names}
{The argument \Arg{piece names} is a comma separated list of names of
  pieces which will \emph{not} be shown when using the \Cmd{showboard}
  commands. Note: called with an empty list no pieces are shown!}


%\newpage
\section{Move arrows}

\command{printarrow}{from,to} 
{Draws an arrow on the last typeset
  board from the square \Arg{from} to the square \Arg{to}.}

\Cmd{highlight}\Arg{[ms]}\Arg{\{square list\}}\newline
The comma separated \Arg{square list} will by default be highlighted using a
  thick frame on the last
  typeset board. The optional marker symbol
\Arg{ms} can be one of X, x, O and o in
  which case a cross or a circle is used to highlight the square.

\command{printknightmove}{from,to}
{Draws a bent arrow from the square \Arg{from} to the square \Arg{to}.}

%\newpage
%%%%%%%%%%%%%%%%%%%%%%%%%%%%%%%%%%%%%%%%%%%%%%%%%%%%%%%%%%%%%%%%%%
\section{Customizations}

\command{newskaklanguage}{language,piecenames}
{Defines a new \Arg{language} for the input of SAN
  moves. \Arg{piecenames} are the uppercase letters used for the
  pieces in the order King, Queen, Rook, Bishop, Knight,
  Pawn. Example: 
\mbox{\texttt{\textbackslash newskaklanguage\{danish\}\{KDTLSB\}}}
defines  \Arg{danish} as a new input language.}

\Cmd{skaklanguage}\Arg{[language]}\newline
Chooses \Arg{language} as new input language --- defaults to \Arg{english}.

\smallskip

\simpleCommand{mainlinestyle}
{Activates the typesetting style for the mainline --- this command can
  be redefined if special requirements for the typesetting exists.}

\simpleCommand{variationstyle}
{Similar to \Cmd{mainlinestyle} just for the typesetting of variations.}


%%%%%%%%%%%%%%%%%%%%%%%%%%%%%%%%%%%%%%%%%%%%%%%%%%%%%%%%%%%%%%%%%%
\section{Game storage}


\command{savegame}{file name}
{Writes the FEN string for the current position on the board to the
  file \texttt{\textless file name\textgreater.fen}}

\command{loadgame}{file name} {Load the position stored in the file
  \texttt{\textless file name\textgreater.fen}}

\command{storegame}{name}
{Stores the current game position using \Arg{name} as reference.}

\command{restoregame}{name}
{Restores the game previously saved using \Cmd{storegame}.} 


%%%%%%%%%%%%%%%%%%%%%%%%%%%%%%%%%%%%%%%%%%%%%%%%%%%%%%%%%%%%%%%%%%
\section{Package options}

\Option{ps}{Includes the \texttt{ps-tricks} package in order to make
  ornaments on the board. Required to make the following commands
  work: \Cmd{showmoverOn}, \Cmd{printarrow}, \Cmd{highlight},
  \Cmd{printknightmove}.}

\Option{psoff}{Does \emph{not} include the \texttt{ps-tricks} package.}

\Option{mover}{Issues the \Cmd{showmoveOn} command.}

\Option{moveroff}{Issues the \Cmd{showmoveOn} command.}

\Option{notation}{Issues the \Cmd{notationOn} command.}

\Option{english}{Makes english the preferred input language --- the
  only defined language at the moment.}

\Option{styleA}{Chooses \Cmd{styleA} style for the typesetting of
  moves.}

\Option{styleB}{Chooses \Cmd{styleB} style for the typesetting of
  moves.}

\Option{styleC}{Chooses \Cmd{styleC} style for the typesetting of
  moves.}

\newcommand{\sizeoption}[1]{\Option{#1}{The board is shown using the
    #1 size font.}}

\sizeoption{tiny}

\sizeoption{small}

\sizeoption{normal}

\sizeoption{large}

\bigskip

The default options are \textsf{\textbf{notation, normal, psoff,
    english, moveroff, styleB}}.

\end{document}






